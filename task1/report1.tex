\documentclass[UTF8,a4paper,11pt]{ctexart}

%\usepackage{ctex} % 中文支持
\usepackage{fancyhdr}
\usepackage{multicol} % 正文单双栏混排
\usepackage{lastpage} % 用于获得最大页数,页眉显示用
\usepackage{geometry} % 用于设置页边距
\usepackage[subfigure,AllowH]{graphfig} %图片相关
\usepackage{graphicx}  %插入图片
\usepackage{amsmath}
\usepackage{amssymb}
\usepackage{booktabs}
\usepackage{float}
\usepackage{subfig}  
\usepackage{multirow}
\usepackage{array}


%geometry使用手册
%http://www.ctex.org/documents/packages/layout/geometry.htm

%\geometry{left=3cm,right=3.8cm,top=2.5cm,bottom=2.5cm}

%定义行间距为1.1倍行距
\renewcommand{\baselinestretch}{2}
%重新定义缩进长度 pt是字号
\parindent 22pt


% 页眉页脚定义
% 因为首页会自动定义成plain格式 http://www.ctex.org/documents/packages/layout/fancyhdr.htm
% but我喜欢每一页都有页眉,so重定义plain型,
% 后面就全设置成plain型好了orz,其实应该改成fancy型再设置fancy的属性

\fancypagestyle{plain}{
\fancyhf{}
%\lhead{Month, Year}
%\chead{\centering{chinese latex template}}
%\rhead{Page \thepage\ of \pageref{LastPage}}
\lfoot{}
\cfoot{}
\rfoot{}}
\pagestyle{plain}


% texbf{…}为加粗
% huge{…}等等调节字体的
\title{第一次作业报告}
\author{姓名:魏宇凡\quad 学号:PB21020576 \quad 日期:2023年3月27日}
\date{} 


\begin{document}

\maketitle


% 中文摘要
% 调整摘要、关键词,中图分类号的页边距
% 中英文同时调整
% 因为geometry命令不能用在正文区只能用这看起来很麻烦的方法了orz

\setlength{\oddsidemargin}{ 1cm} % 3.17cm - 1 inch
\setlength{\evensidemargin}{\oddsidemargin}
\setlength{\textwidth}{13.50cm}

%添加标题和摘要的距离
%vspace{…}是竖直距离
%hspace{…}是水平距离
\vspace{-0.2cm}
%center是居中用的

\vspace{0.5cm}



%%%%%%%%%%%%%%%%%%%%%%%%%%%%%%%%%%%%%%%%%%%%%%%%%%%%%%%%%%%%%%%%
% 正文由此开始-------------------------
%%%%%%%%%%%%%%%%%%%%%%%%%%%%%%%%%%%%%%%%%%%%%%%%%%%%%%%%%%%%%%%%
%%%%%%%%%%%%%%%%%%%%%%%%%%%%%%%%%%%%%%%%%%%%%%%%%%%%%%%%%%%%%%%%
% 恢复正文页边距
%%%%%%%%%%%%%%%%%%%%%%%%%%%%%%%%%%%%%%%%%%%%%%%%%%%%%%%%%%%%%%%%
\setlength{\oddsidemargin}{-.5cm} % 3.17cm - 1 inch
\setlength{\evensidemargin}{\oddsidemargin}
\setlength{\textwidth}{24.00cm}





{\large\textbf{1,任务目标:}}\\
该代码实现了输入计算机科学家姓名,然后可以返回其所有文献标题并展示的功能。\\


{\large\textbf{2,实验细节:}}\\


1,API的使用:\\
本次实验调用了requests与bs4中的BeautifulSoup\ API,其中requests包中的get函数需要str类型的参数,返回requests.models.Response类型,然后通过content再
转换为bytes,接着经过bs4包中的bs4.BeautifulSoup进行处理。\\



2,自定义函数与类:\\
本次实验自定义函数为findweb与getart_name。findweb用于找到与姓名对应的网址,其输入参数为str,即科学家的名字,返回参数也为str,是姓名对应的网址
而getart_name负责记录该网站上所有的文章标题,其输入参数为str,为对应的网址,返回参数为list,其中每个元素都为一个文章的标题。\\


3,实验困难与解决方案:\\
第一个困难就是不熟悉不同API中对应函数的用法。于是通过csdn社区查询语法开始上手。接着就是查询姓名后会返回许多名字相近的科学家的网址,于是需要去判断是哪个符合
要求,所幸该网站的检索是符合程度越高的排在越前面,于是只要返回第一个读到的网址即可



{\large\textbf{3,实验总结:}}\\




{\large\textbf{4,界面布局:}}\\

   
   
\end{document}
